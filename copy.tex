%mscS --> seminar
%msc --> Thesis

% برای گزارش سمینار mscS  و برای پایان‌نامه ,msc را انتخاب کنید
\documentclass[oneside,openany,mscS]{SBU-Thesis}

% برای وارد کردن بسته‌هایی که نیاز است قبل از bidi وارد شوند، آنها را در فایل confix.tex قرار دهید. سایر بسته‌های مورد نیاز را در اینجا می‌توانید وارد کنید:

\begin{document}
	%عنوان پایان‌نامه
	\title{به‌کارگیری رویکرد‌های تطابق معنایی در بازیابی اشیا}
	
	%رشته
	\subject{مهندسی کامپیوتر}
	
	%گرایش
	\field{نرم‌افزار}
	
	%نام
	\name{آرش}
	
	%نام خانوادگی
	\surname{درگاهی نوبری}
	
	%استاد راهنما
	\firstsupervisor{دکتر محمود نشاطی}
	
	%استاد راهنمای دوم (در صورت وجود، در غیر این صورت این خط را حذف کنید)
%	\secondsupervisor{دکتر محمود نشاطی}

	%استاد مشاور (در صورت وجود، در غیر این صورت این خط را حذف کنید)
%	\firstadvisor{دکتر محمود نشاطی}

%تاریخ انجام پایان‌نامه
	\thesisdate{زمستان ۱۳۹۷}
	
	%نام دانشکده
	\faculty{دانشکده مهندسی و علوم کامپیوتر}
	
	%نام دانشگاه
	\university{دانشگاه شهید بهشتی}
	
	%اسامی برای صفحات داوران: در هر مورد، در صورت عدم وجود، کل خط را حذف کنید
	\davaranSupervisor{نام و نام‌خانوادگی}
	\davaranSecondSupervisor{نام و نام‌خانوادگی}
	\davaranAdvisor{نام و نام‌خانوادگی}
	\davaranInternal{نام و نام‌خانوادگی}
	\davaranExternal{نام و نام‌خانوادگی}
	\davaranAssignee{نام و نام‌خانوادگی}
%	\davaranDate{۱۳۹۷/۱۱/۷} % چنانچه این خط را حذف کنید با فضای خالی جایگزین می‌شود

	%مقادیر انگلیسی برای صفحه آخر
	\latinuniversity{Shahid Beheshti University}
	\latinfaculty{Faculty of Computer Science \& Engineering}
	\latintitle{Semantic Matching Methods on Object Retrieval}
	\latinname{Arash}
	\latinsurname{Dargahi Nobari}
	\latinthesisdate{2019}
	\firstlatinsupervisor{Dr. Mahmood Neshati}
	%	\secondlatinsupervisor{Dr. Mahmood Neshati}
	%	\firstlatinadvisor{Dr. Mahmood Neshati}
	% چکیده به فارسی
	\fa-abstract{
		لورم ایپسوم ( به انگلیسی \lr{lorem ipsum} ) متنی بی مفهوم است که تشکیل شده از کلمات معنی دار یا بی معنی کنار هم. کاربر با دیدن متن لورم ایپسوم تصور میکند متنی که در صفحه مشاهده میکند این متن واقعی و مربوط به توضیحات صفحه مورد نظر است واقعی است. حالا سوال اینجاست که این متن « لورم ایپسوم » به چه دردی میخورد و اساسا برای چه منظور و هدفی ساخته شده است؟ پیش از بوجود آمدن لورم ایپسوم ، طراحان وب سایت در پروژه های وب سایت و طراحان کرافیک در پروژه های طراحی کاتولوگ ، بروشور ، پوستر و ... همواره با این مشکل مواجه بودند که صفحات پروژه خود را پیش از آنکه متن اصلی توسط کارفرما ارائه گردد و در صفحه مورد نظر قرار گیرد چگونه پر کنند؟؟ اکثر طراحان با نوشتن یک جمله مانند «این یک متن نمونه است» ویا «توضیحات در این بخش قرار خواهند گرفت» و کپی آن به تعداد زیاد یک یا چند پاراگراف متن میساختند که تمامی متن ها و کلمات ، جملات و پاراگراف ها تکراری بود و از این رو منظره خوبی برای بیننده نداشت و ضمنا به هیچ وجه واقعی به نظر نمیرسید تا بتواند شکل و شمایل تمام شده پروژه را نشان دهد. از این رو متنی ساخته شد که با دو کلمه ( به فارسی : لورم ایپسوم ) آغاز میشد وبا همین نام در بین طراحان وب و گرافیک شناخته و به سرعت محبوب شد. وب سایت های سازنده لورم ایپسوم میتوانند هر تعداد کلمه و پاراگراف که بخواهید به صوورت تکراری یا غیر تکراری برایتان بسازند و تحویلتان بدهند تا از آنها در پروژه هایتان استفاده کنید. ( لورم ایپسوم فارسی) متن های لورم ایپسوم را به زبان فارسی و علاوه بر زبان فارسی به انگلیسی ، عربی ، ترکی استانبولی و ... برایتان میسازد. زبان های دیگر نیز رفته رفته به بانک اطلاعاتی لورم ایسپوم فارسی اضافه خواهند شد.  
	}
	%کلمات کلیدی:
	\keywords{واژه۱، واژه۲، واژه۳}
	%%%%%%%%%%%%%%%%%%%%%%%%%%%%%%%%%
	
	
\firstPage %ساخت صفحه اول پایان‌نامه 
\davaranPage % ساخت صفحه امضا داوران

%%%%%%ساخت صفحه تقدیر و تشکر%%%%%%%%%%
{
	\newpage
	\thispagestyle{plain}
	\noindent
	\large{\textbf{با سپاس و قدردانی از}}
	
	\noindent
	پدران و مادرانی که خود را فدای تربیت فرزاندان خود کردند و\\
	اساتید و معلمانی که در تمام دوران زندگی، راهنمای جانسوز ما بودند.
	
	
	\vspace{14cm}	
	آوردن این صفحه اختیاریست.
	
	\pagebreak
}
%%%%%%%%%%%%%%%%%%%%%%%%%%%%%%%%

\rightsPage % ساخت صفحه حقوق پایان‌نامه
\copyRightPage % ساخت صفحه تعهدنامه

%%%%%%ساخت صفحه تقدیم به %%%%%%%%%%
{
	\newpage
	\thispagestyle{plain}
	\large{\textbf{تقدیم به}}
	
	\begin{center}
		رهجويان علم و فناوری و دوست‌داران علم و دانش
	\end{center}
	
	\vspace{14cm}	
	آوردن این صفحه اختیاریست.
	
	\pagebreak
}


%%%%%%%%%%%%%%%%%%%%%
\abstractPage % ساخت صفحه چکیده


\tableofcontents % فهرست مطالب
\listoffigures \newpage % فهرست تصاویر
\listoftables \newpage % فهرست جداول

		

%%%%%%%%%%%%%%%%%%%%%%%%%%%

\include{body}

%%%%%%%%%%%%%%%%%%%%%%%%%%%
	

% مراجع
\newpage
%\onehalfspacing
\bibliographystyle{ieeetr-fa}%{chicago-fa}%{plainnat-fa}%
\bibliography{thesis}


%%%%%%%%%% نمونه‌ی واژه‌نلمه تک ستونه ٪٪٪٪٪٪٪٪٪٪
\baselineskip=.6cm

\chapter*{واژه‌نامه  انگلیسی به  فارسی}\markboth{واژه‌نامه  انگلیسی به  فارسی}{واژه‌نامه  انگلیسی به  فارسی}
%\thispagestyle{fancy}
\addcontentsline{toc}{chapter}{واژه‌نامه  انگلیسی به  فارسی}
\noindent
\englishgloss{Word}{کلمه}
\englishgloss{Word 2}{کلمه ۲}
\englishgloss{Word 3}{کلمه ۳}


\chapter*{واژه‌نامه فارسی به انگلیسی}\markboth{واژه‌نامه فارسی به انگلیسی}{واژه‌نامه فارسی به انگلیسی}
\addcontentsline{toc}{chapter}{واژه‌نامه فارسی به انگلیسی}
%\thispagestyle{fancy}
\noindent
\persiangloss{کلمه ۱}{word 1}
\persiangloss{کلمه ۲}{word 2}
\persiangloss{کلمه ۳}{word 3}

\baselineskip=1cm
%%%%%%%%%%%%%%%%%%%%%%%%%


% چکیده به انگلیسی
\en-abstract
{
	This is Abstract in English.
}

% کلمات کلیدی انگلیسی 
\latinkeywords
{
	Word1, Word2, Word3
}

\latinAbstractPage % ساخت صفحه چکیده به انگلیسی
\latinFirstPage % ساخت صفحه آخر

	
\end{document}