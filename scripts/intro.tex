\cchapter{مقدمه}
اینترنت اشیا\LTRfootnote{Internet of Things} یکی از فناوری‌های مورد توجه این روزها در دنیای کامپیوتر است. یکی از عواملی که سبب احساس نیازمندی به این فناوری نوپا گشته، اهمیت یافتن استفاده از اینترنت در دهه اخیر می‌باشد. امروزه نیازمندی انسان به اتصال و بهره‌مندی از شبکه جهانی اینترنت بر کسی پوشیده نیست و حتی بسیاری از نیازهای ابتدایی انسان از طریق اینترنت مرتفع می‌گردد. در چنین شرایطی، برای استفاده از انواع ماشین‌هایی که پیش از این نقش به‌سزایی در بهبود سطح زندگی انسان ایفا می‌کرده‌اند؛ نیازمندی جدیدی حس می‌شد که آن، فعالیت این ماشین‌ها بر بستر اینترنت بود. این امر باعث فعالیت یکپارچه‌تر دستگاه‌هایی می‌شد که پیش از این به طور مستقل از یکدیگر نیازمندی‌های بشر را تامین می‌کردند و بدین سبب، مفهوم اینترنت اشیا معرفی گردید.\\
با گسترش روزافزون نیازمندی‌های مبتنی بر اینترنت اشیا، نیازمندی‌ها و همچنین کاربردهای جدیدی برای این فناوری کشف می‌گردد. برای مثال امروزه شاهد استفاده از سیستم‌های مبتنی بر اینترنت اشیا در کاربرد‌هایی گوناگون همچون ساختمان‌های هوشمند، شهرهای هوشمند و مواردی از این قبیل هستیم. به عنوان مثال، در یک سیستم مبتنی بر اینترنت اشیا که قرار است مدیریت یک شهر هوشمند را برعهده داشته باشد، هزاران حسگر\LTRfootnote{Sensor} در محیطی گسترده (در مقیاس کیلومتر مربع) مشغول به فعالیتند. در چنین شرایطی دو عامل اصلی به عنوان عوامل مورد بررسی جهت سنجش قابل اعتماد بودن\LTRfootnote{Reliability} سیستم مطرح است. این دو عامل دقت مشاهده\LTRfootnote{Observation} محیط و همچنین میزان انرژی مصرفی محیط می‌باشند. در ادامه به بررسی این دو عامل و رابطه‌شان نسبت به یکدیگر می‌پردازیم.\\
یکی از عوامل ذکر شده در بالای برای بررسی میزان قابل اعتماد بودن سیستم‌های مبتنی بر اینترنت اشیا، دقت مشاهده محیط است. فرض کنید در یک خانه هوشمند، قرار است سیستم مراقبت از سالمندان در آن پیاده‌سازی شود. این سیستم باید به طور مداوم به مشاهده محیط پرداخته و در صورت بروز هرگونه رفتار غیرطبیعی در سالمند (از جمله افتادن بر روی زمین و...) اورژانس را از وضعیت به وقوع پیوسته مطلع سازد. علاوه بر این اطلاعاتی از قبیل فشار خون، تعداد ضربان و... نیز از شخص سالمند که در طول چند ساعت اخیر ضبط شده است به سیستم درمانی ارائه دهد. مشخص است که در چنین شرایطی و در چنین سیستمی اهمیت عامل اول، یعنی دقت سیستم بسیار زیاد است.\\
برای دستیابی به چنین هدفی لازم است تا حسگرها به طور مداوم و با فواصل زمانی کم به مشاهده محیط پرداخته و اطلاعات موردنیاز را از محیط جمع‌آوری نمایند. بدین ترتیب، اطلاعات به‌دست‌آمده از محیط، اطلاعات به‌روزتری خواهد بود و به همین دلیل دقت سیستم افزایش می‌یابد. البته به غیر از تناوب مشاهده محیط توسط حسگر‌ها، عوامل دیگری همچون کیفیت خود حسگرها، نحوه نصب آن‌ها در محیط و سایر موارد نیز می‌تواند تاثیر به‌سزایی در دقت اندازه‌گیری مقادیر در محیط و در نتیجه مشاهده بهتر آن داشته باشد که البته این موارد در حیطه بحث این پژوهش نخواهد بود.\\
اما عامل دیگر در بررسی سیستم‌های مبتنی بر اینترنت اشیا که اتفاقا مبحث اصلی این پژوهش نیز بوده، مصرف انرژی\LTRfootnote{Energy consumption} است. یک سیستم مبتنی بر اینترنت اشیا، شامل چندین حسگر است که در محیط مستقر بوده و به تناوب در حال دریافت اطلاعات محیط می‌باشند. بدیهی‌ست که دو موضوع در بحث تامین انرژی حسگرهای چنین سیستمی مطرح است. اول اینکه اگر حسگرهای این سیستم به‌طور بی‌رویه به مشاهده محیط و دریافت اطلاعات از آن بپردازند، انرژی بسیار زیادی مصرف می‌شود که خود این امر تا حدود بسیاری نامطلوب است. به‌هرحال در دنیای امروز یکی از مباحث بسیار مهم و حائز اهمیت، بحث مصرف بی‌رویه منابع انرژی است و واضح است که در صورت بهینه نبودن سیستم‌های مبتنی بر اینترنت اشیا به لحاظ مصرف منابع انرژی، این سیستم‌ها نخواهند توانست جای خود را در زندگی روزمره ما بیابند.\\
علاوه بر این، موضوع دیگری که اهمیت مصرف بهینه منابع انرژی را برای سیستم‌های مبتنی بر اینترنت اشیا دو چندان می‌کند، بحث طول عمر این سیستم‌ها است. با توجه به ماهیت سیستم‌های مبتنی بر اینترنت اشیا، معمولا این سیستم‌ها به لحاظ گستره جغرافیایی سیستم‌هایی بسیار بزرگ هستند که گاها مساحت تحت پوشش آن‌ها می‌تواند به بزرگی یک شهر باشد (همانند مثال شهر هوشمند). بنابراین آنچه حائز اهمیت است، این است که امکان تامین انرژی تمام سیستم به صورت متمرکز ممکن است وجود نداشته باشد. از این روی، رویکرد اصلی در این سیستم‌ها این است که حسگرها خود توانایی تامین انرژی خود را به طور مستقل داشته باشند. یکی از راهکارها در این زمینه، استفاده از باتری برای تامین انرژی حسگرهاست که در جهت مستقل بودن حسگرها و همچنین افزایش قابلیت جابه‌جایی\LTRfootnote{Portability} آن‌ها صورت می‌گیرد. بنابراین منابع انرژی که در اختیار هر حسگر قرار می‌گیرد، محدود است و استفاده بهینه از این منابع بسیار اهمیت می‌یابد. هر چند این منابع قابل تمدید هستند (یعنی می‌توان باتری یک حسگر را با باتری جدید معاوضه نمود.) اما آنچه حائز اهمیت است، در واقع فعال نگه داشتن سیستم برای مدت طولانی تری است. به طور کلی می‌توان طول عمر سیستم را یکی از مزایای بهینه‌سازی مصرف منابع انرژی دانست که منجر به قابل اعتماد بودن بیشتر سیستم می‌گردد.\\
بنابراین همان‌گونه که مشخص است یک \lr{Trade-Off} در بین دقت عملکرد سیستم و طول عمر سیستم وجود دارد. به این معنی که اگر بخواهیم دقت سیستم را افزایش دهیم، حسگرها باید با تناوب‌های زمانی کوتاه‌تری به مشاهده و دریافت اطلاعات محیط بپردازند و در نتیجه انرژی بیشتری مصرف کنند که این امر منجر به کاهش طول عمر حسگرها و در نهایت کاهش طول عمر سیستم می‌گردد. و در طرف مقابل اگر در صدد افزایش طول عمر سیستم‌های مبتنی بر اینترنت اشیا باشیم، بدیهی است که باید تناوب زمانی فعالیت حسگرها طولانی‌تر گردد. این امر باعث می‌شود تا دریافت اطلاعات از محیط با فواصل زمانی بیشتری از جانب حسگرها صورت پذیرد و در نتیجه دقت اطلاعات حاصله از محیط کاهش یابد. چرا که ممکن است تغییراتی در محیط، در فواصل زمانی‌ای رخ دهد که حسگرها مشغول به فعالیت نیستند و در نتیجه برخی اطلاعات در ارتباط با رویدادهای رخ داده در محیط از دست بروند.\\
همان‌طور که مشاهده شد، یک تضاد در میان دو عامل اصلی جهت به‌دست‌آوردن اعتماد برای استفاده از سیستم‌های مبتنی بر اینترنت اشیا وجود دارد. آن‌چه در این تحقیق به آن پرداخته شده است در واقع پدید آوردن توازنی نسبی در میان این دو عامل است. به طوری که یک سیستم مبتنی بر اینترنت اشیا هم دارای دقت قابل قبولی در مشاهده و دریافت اطلاعات مربوط به رویدادهای رخ داده در محیط باشد؛ و هم میزان مصرف منابع انرژی آن بهینه باشد و اتلاف منابع انرژی را به حداقل برساند. برای این منظور پژوهش‌هایی پیش از این صورت گرفته است که ما در این تحقیق به آن‌ها می‌پردازیم. اما به طور کلی چند رویکرد کلی در دستیابی به این امر وجود دارد که در ذیل به شرح مختصری از هر کدام اکتفا می‌کنیم و در بخش‌های بعدی به تفصیل آن‌ها را بررسی می‌نماییم.\\
یکی از رویکردهای پرتکرار در میان مقالات، استفاده از زمانبندی برای فعالیت حسگرهاست. به این معنی که زمان‌های مورد نیاز برای فعالیت یک حسگر مشخص می‌شود و سپس حسگر در این زمان‌ها شروع به فعالیت می‌کند. پس از آن که طبق برنامه زمانبندی شده، زمان فعالیت حسگر به پایان رسید حسگر غیرفعال شده و تا بازه‌ی زمانی بعدی که باید فعالیتش را شروع کند مصرف انرژی خود را به حداقل می‌رساند. این امر اغلب با غیرفعال شدن مدار ارتباطی حسگر صورت می‌گیرد؛ چراکه بخش زیادی از منابع انرژی مصرفی توسط یک حسگر، صرف برقراری ارتباط با دیگر اجزای سیستم می‌گردد. بدیهی است که یکی از دلایل این مصرف زیاد برای برقراری ارتباط این است که اغلب ارتباطات در سیستم‌های مبتنی بر اینترنت اشیا به صورت بی‌سیم\LTRfootnote{Wireless} و با استفاده از تجهیزات ارتباطی نظیر \lr{Bluetooth} صورت می‌پذیرد. در مقالات مختلف دستیابی به یک سیستم زمانبندی برای فعالیت حسگرها، به کمک ابزارهای گوناگون از جمله ارائه معماری برای سیستم‌های مبتنی بر اینترنت اشیا و یا استفاده از مکانیزم های از پیش تعریف شده تحقق یافته است.\\
یکی دیگر از رویکردهای ارائه شده در مقالات مورد بررسی، بحث ارائه یک ساختار جهت برقراری ارتباط میان حسگرها و سایر اجزای سیستم است به طوری که ارتباط آن‌ها با هزینه کمتری به لحاظ مصرف انرژی صورت گیرد. این امر به کمک پیدا کردن مسیرهای کوتاه‌تر در میان اجزای سیستم میسر می‌شود و در واقع اساس کار این رویکردها این است که ارتباط میان اجزای نزدیک به یکدیگر با صرف انرژی کمتری صورت می‌پذیرد. همچنین در این رویکرد، راهکاری برای تجمیع اطلاعاتی که قرار است مورد مخابره قرار گیرد در نظر گرفته شده که از مصرف انرژی بی‌رویه در جهت انتقال اطلاعات در بستر شبکه اینترنت اشیا جلوگیری می‌نماید.\\
در مجموع بعد از بررسی راهکارهای ارائه شده در این پژوهش‌ها مشاهده می‌شود که مصرف انرژی در سیستم‌های مبتنی بر اینترنت اشیا تا مقدار قابل توجهی کاهش یافته و به طور مثال میزان انرژی مصرفی یک سیستم مبتنی بر اینترنت اشیا با ۴۵۰۰ حسگر پس از ۱۲۰ ساعت فعالیت به عددی در حدود $10 \times 10^6 nJ$ کاهش یابد که این مقدار حتی در قیاس با برخی روش‌های دیگر بهینه‌سازی چیزی در حدود ۱/۴ انرژی مصرفی را مورد استفاده قرار می‌دهد.\\
در بخش بعدی گزارش به مروری بر ادبیات تحقیق پرداخته و همچنین اصطلاحات مورد نیاز و همچنین پیش‌زمینه‌های لازم جهت درک بهتر راهکارهای ارائه شده را تشریح می‌نماییم. پس از آن در بخش سوم، به تفصیل، راهکارهای ارائه شده در مقالات مورد بررسی را تشریح کرده و با درخت موضوعی مبحث آشنا خواهیم شد تا ذهنیت مورد نیاز برای زمینه تحقیق را درک نماییم. پس از آن در بخش چهارم به نتیجه‌گیری پرداخته و راهکارهای ارائه شده را با هم مقایسه می‌نماییم و به بیان نقاط ضعف و قوت هریک می‌پردازیم.












